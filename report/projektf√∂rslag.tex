# Projektförslag

- Vilka personer som är med i gruppen (3-4 personer, för- och efternamn)

  Sebastian Gustafsson, Albin Hjelm & Alexander Backlund

- Vad projektet heter/ert gruppnamn (kort och kärnfullt) och vilket nummer er grupp har i Studentportalen

  Positioning in ABW, Groupnr 4

- En kortfattad beskrivning av vad ni ska göra (några rader)

    Att med hjälp av funktionalitet i "Skype for business" kunna bestämma en medarbetares position i en aktivitetsbaserad arbetsplatsmiljö. För att uppnå detta vill vi designa en applikation som är tänkt att användas av administratörer för "Skype for business". Denna applikation ska kunna sniffa information om intilliggande accesspunkter och koppla dessa till en arbetsyta i den aktivitetsbaserade verksamheten. Därefter ska man sedan kunna populera "Skype for business" databas med denna information. Därigenom blir det möjligt att se vart en medarbetare befinner sig i "Skype for business".
    Precisionsmässigt är avsikten att kunna bestämma dels på vilken adress en medarbetare befinner sig, dels på vilket våningsplan och även vart på våningsplanet.
    Precisionen på ett våningsplan bör vara så pass god att den anger ett område med max 10 meters radie.

- Vilka kunskaper från tidigare kurser projektet integrerar
  Datorkommunikation & Distribuerade System, IOOPM,


- Kortfattat hur projektet innebär ett icke-trivialt tekniskt utvecklingsarbete.
  Vi skall utveckla en mobil-applikation, samt använda oss av kunskaper inom nätverkskommunikation.


- Vilka tekniska utmaningar kan ni se redan nu? (Ex: Vad behöver ni lära er, vad kan ni redan? Vad är svårt/lätt? Vad finns färdigt, vad behöver ni själva tillföra?)
  - Vi behöver lära oss att utveckla en android-applikation
  - Hantera felmarginaler som uppstår med all typ av inomhuspositionering
  - Android studio ger mycket färdigbyggd funktionalitet
  - Skype for business tillhandahåller en del användbar funktionalitet
  - En användarvänlig apllikation som genererar den information som behövs för att populera databasen.
  - Vi har baskunskaper för att kunna designa applikation och ta fram algoritmerna för att bestämma postionen.



- Kortfattat hur projektet löser en instans av ett större problem, d.v.s., hur det inte bara är specifikt för just er uppdragsgivare/situation utan även är intressant i ett vidare perspektiv. (Ex: specifikt: visa vägen till målet för MazeMap-användare; generellt: vägval/orientering/kartvisning...)

  Vårat projekt kommer erbjuda ett smidigt sätt för företag att övergå till en aktivitetsbaserad verksamhet med hjälp av "Skype for business" i och med att man kommer kunna lokalisera medarbetare utan fast arbetsplats.

- Kortfattat hur projektet spelar roll (varför problemet ni valt är viktigt att lösa). (Ex: hur underlättar ert projekt vardagen för användarna? Hur förbättras samhällsfunktionen? Vad kan andra göra när ni är klara med er prototyp? etc)

  Vårat projekt spelar roll eftersom det löser ett problem med aktivitetsbaserad verksamhet i och med att medarbetare nu enkelt kommer kunna lokalisera varandra.

- Hur projektet är självständigt, d.v.s. argumentera för att ni inte bara utför ett uppdrag utan att "tänka själva"

  Projektet är självständigt eftersom vi kommer designa ett eget system som underlättar för alla som använder "Skype for business" och vill övergå till en aktivitetsbaserad verksamhet. Vad vi vet existerar inget liknande system i dagsläget.

- Vilket språk (svenska/engelska) ni planerar att skriva rapporten på. Välj engelska bara om ni känner er någorlunda säkra. Är ni osäkra, kontakta gärna Språkverkstaden.

  Svenska.
